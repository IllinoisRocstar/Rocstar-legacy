\setcounter{figure}{0}
\setcounter{table}{0}
\setcounter{equation}{0}

\irsection{Introduction}{Intro}

\Rocburn\ is defined by one-dimensional physics, appropriate for a homogeneous propellant, and it is assumed that its parameters can be chosen to represent the heterogeneous case in some fashion. The physics that is not accounted for in any plausible way in this strategy includes the non-planar surface regression and the diffusion flames in the combustion field. \Rocburn\ is a numerical module inside \Rocstar\ that simulates the thermal transient effects in the wall-normal direction of a homogeneous medium found in solid propellant rocket motors (e.g., insulated walls, propellant, nozzle, etc.). The time-dependent temperature in the solid is computed using a one-dimensional model that takes into account a temperature gradient normal to (only) the burning surface. Thus, the solid is essentially treated as a semi-infinite solid, and the unsteady heat equation is solved in the wall-normal direction. The \Rocburn\ table feature is created by averaging the solid phase heat conduction over a plane that is parallel to the mean surface, accounting for rotational terms that arise due to the uneven surface. Boundary conditions are applied at the propellant surface and include, most importantly, the heat flux from the combustion field. This heat flux is not modeled but is contained in a universal lookup generated by another module in \Rocstar. Information about combustion and ignition of solid propellants is available elsewhere [\cite{Ibiricu:1975,Blomshield:1997,Ward:1998,Brewster:2000,Chen:2002,Jackson:2002,Massa:2002,Massa:2004,Massa:2005}].